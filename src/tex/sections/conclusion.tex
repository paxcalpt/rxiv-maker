\section{Conclusion}

This article demonstrates the successful implementation of Article-Forge, a comprehensive template engine for scientific publications using the HenriquesLab style. The template provides researchers with a robust, automated solution for creating publication-quality documents while minimizing technical barriers and ensuring consistent presentation.

\subsection{Key Contributions}

The main contributions of Article-Forge are:

\begin{enumerate}
    \item \textbf{Comprehensive Template Ecosystem}: A complete LaTeX framework that includes style files, build automation, and documentation infrastructure
    \item \textbf{Self-Documenting Design}: A template that serves as both a tool and its own documentation, demonstrating all features through practical implementation
    \item \textbf{Modern Workflow Integration}: Support for version control, containerization, and automated build processes that align with contemporary research practices
\end{enumerate}

\subsection{Practical Implications}

These findings have several practical implications:
\begin{itemize}
    \item Researchers can focus on content creation rather than document formatting
    \item Research groups can maintain consistent presentation across all publications
    \item Collaborative writing becomes more efficient through modular document structure
    \item Reproducible document generation ensures consistent results across different environments
\end{itemize}

\subsection{Final Remarks}

In conclusion, Article-Forge represents a significant advancement in scientific document preparation tools. By combining professional typography with modern automation practices, the template reduces barriers to high-quality scientific publishing. The self-documenting approach demonstrated in this article provides clear evidence of the template's capabilities and serves as a practical guide for adoption.

The template's success in generating this publication validates its design principles and demonstrates its readiness for widespread use in scientific research communities.

\section*{Acknowledgments}

We acknowledge the development of the original HenriquesLab style and the broader LaTeX community for providing the foundation upon which this template builds. We also recognize the importance of tools like NanoPyX \cite{nanopyx2024} in advancing scientific computing and reproducible research practices.
